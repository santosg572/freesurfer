\documentclass{article}
\usepackage{lipsum}
\usepackage{xcolor}
\begin{document}

\textbf{Methodological considerations on tract-based spatial statistics (TBSS)}

\hfill

Having gained a tremendous amount of popularity since its introduction in 2006, tract-based spatial 
statistics (TBSS) can now be considered as the standard approach for voxel-based analysis (VBA) of 
diffusion tensor imaging (DTI) data. Aiming to improve the sensitivity, objectivity, and interpretability 
of multi-subject DTI studies, TBSS includes a skeletonization step that alleviates residual image 
misalignment and obviates the need for data smoothing. Although TBSS represents an elegant and 
user-friendly framework that tackles numerous concerns existing in conventional VBA methods, it has 
limitations of its own, some of which have already been detailed in recent literature. In this work, we 
present general methodological considerations on TBSS and report on pitfalls that have not been described 
previously. In particular, we have identified specific assumptions of TBSS that may not be satisfied under 
typical conditions. Moreover, we demonstrate that the existence of such violations can severely affect the 
reliability of TBSS results. With TBSS being used increasingly, it is of paramount importance to acquaint 
TBSS users with these concerns, such that a well-informed decision can be made as to whether and how to 
pursue a TBSS analysis. Finally, in addition to raising awareness by providing our new insights, we provide 
constructive suggestions that could improve the validity and increase the impact of TBSS drastically.

{\color{blue} }


 \end{document}


