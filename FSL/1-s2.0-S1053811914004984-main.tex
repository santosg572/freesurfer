\documentclass{article}
\usepackage{lipsum}
\usepackage{xcolor}
\begin{document}

\textbf{Methodological considerations on tract-based spatial statistics (TBSS)}

\hfill

Having gained a tremendous amount of popularity since its introduction in 2006, tract-based spatial 
statistics (TBSS) can now be considered as the standard approach for voxel-based analysis (VBA) of 
diffusion tensor imaging (DTI) data. Aiming to improve the sensitivity, objectivity, and interpretability 
of multi-subject DTI studies, TBSS includes a skeletonization step that alleviates residual image 
misalignment and obviates the need for data smoothing. Although TBSS represents an elegant and 
user-friendly framework that tackles numerous concerns existing in conventional VBA methods, it has 
limitations of its own, some of which have already been detailed in recent literature. In this work, we 
present general methodological considerations on TBSS and report on pitfalls that have not been described 
previously. In particular, we have identified specific assumptions of TBSS that may not be satisfied under 
typical conditions. Moreover, we demonstrate that the existence of such violations can severely affect the 
reliability of TBSS results. With TBSS being used increasingly, it is of paramount importance to acquaint 
TBSS users with these concerns, such that a well-informed decision can be made as to whether and how to 
pursue a TBSS analysis. Finally, in addition to raising awareness by providing our new insights, we provide 
constructive suggestions that could improve the validity and increase the impact of TBSS drastically.

{\color{blue} Tras haber ganado una enorme popularidad desde su introducción en 2006, la estadística 
espacial basada en tractos (TBSS) puede considerarse ahora el enfoque estándar para el análisis basado en 
vóxeles (VBA) de datos de imágenes de tensor de difusión (DTI). Con el objetivo de mejorar la 
sensibilidad, la objetividad y la interpretabilidad de los estudios de DTI multisujeto, TBSS incluye un 
paso de esqueletización que alivia la desalineación residual de la imagen y elimina la necesidad de 
suavizar los datos. Si bien TBSS representa un marco elegante y fácil de usar que aborda numerosas 
preocupaciones existentes en los métodos de VBA convencionales, presenta sus propias limitaciones, algunas 
de las cuales ya se han detallado en la literatura reciente. En este trabajo, presentamos consideraciones 
metodológicas generales sobre TBSS e informamos sobre las dificultades que no se han descrito previamente. 
En particular, hemos identificado supuestos específicos de TBSS que podrían no cumplirse en condiciones 
típicas. Además, demostramos que la existencia de tales violaciones puede afectar gravemente la fiabilidad 
de los resultados de TBSS. Dado el creciente uso de TBSS, es fundamental familiarizar a los usuarios de 
TBSS con estas inquietudes, para que puedan tomar una decisión informada sobre si realizar un análisis 
TBSS y cómo hacerlo. Finalmente, además de generar conciencia con nuestros nuevos conocimientos, ofrecemos 
sugerencias constructivas que podrían mejorar la validez y aumentar drásticamente el impacto de TBSS.}


 \end{document}


