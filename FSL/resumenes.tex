\documentclass[12pt]{article}
\begin{document}

\textbf{Advantages of QBI in TBSS analyses}

\hfill

Diffusion-weighted magnetic resonance imaging (DWMRI) is used to study white matter (WM) in normal and 
clinical populations. In DWMRI studies, diffusion tensor imaging (DTI) models the WM anisotropy with one dominant direction, detecting possible pathway abnormalities only in large and highly coherent fiber tracts. However, more general anisotropy models like Q-ball imaging (QBI) may provide more sensitive WM descriptors in single patients. The present study aimed to compare DTI and QBI models in a group-level population analysis, using Amyotrophic Lateral Sclerosis (ALS) as a pathological case model of WM tract degeneration.

DWMRI was performed in 19 ALS patients and 19 age and sex-matched healthy controls. DTI and QBI estimates 
were compared in whole-brain tract-based spatial statistics (TBSS) and volume of interest (VOI) analyses, 
and correlated with ALS clinical scores of disability.

A significant decrease of the QBI-derived generalized fractional anisotropy (GFA) was observed in both 
motor and extramotor fibers of ALS patients compared to controls. Homologue DTI-derived FA maps were only 
partially overlapping with GFA maps. Particularly, the left corticospinal tracts resulted more markedly 
depicted by the QBI than by the DTI model, with GFA predicting ALS disability better than FA.

The present findings demonstrate that QBI model is suitable for studying WM tract degeneration in 
population-level clinical studies. Particularly, group-level studies of fiber integrity may benefit from 
QBI when DTI is biased towards low values, such as in cases of fiber degeneration, and in regions with more 
than one dominant fiber direction.

\par\rule{\textwidth}{0.5pt} 


\textbf{Advances in functional and structural MR image analysis and implementation as FSL}

\hfill

The techniques available for the interrogation and analysis of neuroimaging data have a large influence in 
determining the flexibility, sensitivity, and scope of neuroimaging experiments. The development of such 
methodologies has allowed investigators to address scientific questions that could not previously be 
answered and, as such, has become an important research area in its own right.

In this paper, we present a review of the research carried out by the Analysis Group at the Oxford Centre 
for Functional MRI of the Brain (FMRIB). This research has focussed on the development of new methodologies 
for the analysis of both structural and functional magnetic resonance imaging data. The majority of the 
research laid out in this paper has been implemented as freely available software tools within FMRIB’s 
Software Library (FSL).

\par\rule{\textwidth}{0.5pt}


\textbf{Bayesian analysis of neuroimaging data in FSL}   

\hfill

Typically in neuroimaging we are looking to extract some pertinent information from imperfect, noisy images 
of the brain. This might be the inference of percent changes in blood flow in perfusion FMRI data, 
segmentation of subcortical structures from structural MRI, or inference of the probability of an 
anatomical connection between an area of cortex and a subthalamic nucleus using diffusion MRI. In this 
article we will describe how Bayesian techniques have made a significant impact in tackling problems such 
as these, particularly in regards to the analysis tools in the FMRIB Software Library (FSL). We shall see 
how Bayes provides a framework within which we can attempt to infer on models of neuroimaging data, while 
allowing us to incorporate our prior belief about the brain and the neuroimaging equipment in the form of 
biophysically informed or regularising priors. It allows us to extract probabilistic information from the 
data, and to probabilistically combine information from multiple modalities. Bayes can also be used to not 
only compare and select between models of different complexity, but also to infer on data using committees 
of models. Finally, we mention some analysis scenarios where Bayesian methods are impractical, and briefly 
discuss some practical approaches that we have taken in these cases.


\end{document}

